\subsection{Liberté}\label{libertuxe9}

\begin{quote}
\hyperlink{donatella-della-ratta}{Donatella
Della Ratta}
\end{quote}

Sur mes cahiers d'écolier Sur mon pupitre et les arbres Sur le sable sur
la neige J'écris ton nom

Sur toutes les pages lues Sur toutes les pages blanches Pierre sang
papier ou cendre J'écris ton nom

Sur les images dorées Sur les armes des guerriers Sur la couronne des
rois J'écris ton nom

Sur la jungle et le désert Sur les nids sur les genêts Sur l'écho de mon
enfance J'écris ton nom

Sur les merveilles des nuits Sur le pain blanc des journées Sur les
saisons fiancées J'écris ton nom

Sur tous mes chiffons d'azur Sur l'étang soleil moisi Sur le lac lune
vivante J'écris ton nom

Sur les champs sur l'horizon Sur les ailes des oiseaux Et sur le moulin
des ombres J'écris ton nom

Sur chaque bouffée d'aurore Sur la mer sur les bateaux Sur la montagne
démente J'écris ton nom

Sur la mousse des nuages Sur les sueurs de l'orage Sur la pluie épaisse
et fade J'écris ton nom

Sur les formes scintillantes Sur les cloches des couleurs Sur la vérité
physique J'écris ton nom

Sur les sentiers éveillés Sur les routes déployées Sur les places qui
débordent J'écris ton nom

Sur la lampe qui s'allume Sur la lampe qui s'éteint Sur mes maisons
réunies J'écris ton nom

Sur le fruit coupé en deux Du miroir et de ma chambre Sur mon lit
coquille vide J'écris ton nom

Sur mon chien gourmand et tendre Sur ses oreilles dressées Sur sa patte
maladroite J'écris ton nom

Sur le tremplin de ma porte Sur les objets familiers Sur le flot du feu
béni J'écris ton nom

Sur toute chair accordée Sur le front de mes amis Sur chaque main qui se
tend J'écris ton nom

Sur la vitre des surprises Sur les lèvres attentives Bien au-dessus du
silence J'écris ton nom

Sur mes refuges détruits Sur mes phares écroulés Sur les murs de mon
ennui J'écris ton nom

Sur l'absence sans désirs Sur la solitude nue Sur les marches de la mort
J'écris ton nom

Sur la santé revenue Sur le risque disparu Sur l'espoir sans souvenir
J'écris ton nom

Et par le pouvoir d'un mot Je recommence ma vie Je suis né pour te
connaître Pour te nommer

\textbf{Liberté.} -- Paul Éluard, ``Liberté'' from ``Poésies et
vérités'', 1942\footnote{Editor's note: we could not provide the English
  translation of the poem of Paul Eluard because it's copyrighted and
  the translator didn't want to participate. Therefore we provide the
  original French version which is in the Public Domain.}

Paul Eluard wrote this poem called ``Liberté'' in the darkest moment of
the world's history, during World War II. At the time, France was
occupied by Nazis. Violence, destruction, death were everywhere. The
world was hopeless and with a very dark future ahead.

Yet words of hope were still alive, and poets such as Eluard were able
to give shapes and sounds to these words. Freedom, Liberté.

Libertà, Hurriyya. I've never heard such a beautiful word being spoken
by a human voice as much as when I heard this word resonating in the
country I love the most, Syria.

It was the chant of life. It was about people telling the world, telling
themselves: we are a-l-i-v-e!\\
You, my dear friend, have taken this word well above the cage of silence
where it was exiled.\\
You, the Syrian youth, many of our friends who were rushing to write
``freedom'' on the city's walls, everywhere, are now paying the price.

It's dark time, just like when Paul wrote his ode to liberté. It's dark
time, my friend, and that's why we need you and your words more than
ever.

That sweet sound, the sound of spring coming, the sound of youth..will
come back..with you, my friend.\\
Hurriyya.
