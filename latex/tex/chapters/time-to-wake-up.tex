\subsection{Time to Wake Up}\label{time-to-wake-up}

\begin{quote}
\href{../appendix/attributions.html\#mushon-zer-aviv}{Mushon Zer-Aviv}
\end{quote}

Freedom of information, much like freedom of markets, doesn't
``naturally'' lead to the kind of freedom we hope for in society. In
fact, in the past decade since the rise of the free culture movement,
we've seen many costs such as time, attention and education shifting to
the side of content creators while financial profit is centralized by
the data-hoarding Internet giants that enjoy the reputation of
information liberators. Google, for example, is considered a great
patron for free culture, whereas in practice it cannibalizes the free
culture that it monetizes, offsetting the costs of culture from those
consuming it, and profiting from those creating it, and that's us.

The technological principle that powers digital freedom of information,
and that we celebrate through free culture and the creative explosion of
the web, is the same technological principle that powers digital
surveillance. We have to stop seeing these technological principles as
``ready-made for culture'' whether that be as a pre-made model for
cultural exchange, or as a pre-made model for the end of privacy. This
techno-determinism is a double-edged sword; it's time to wake up and
realize that the new possibilities and challenges posed by digital
networks should inform the way we decide to live our lives, not dictate
it.
